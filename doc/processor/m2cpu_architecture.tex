\documentclass[a4paper,12pt]{article}

\PassOptionsToPackage{hyphens}{url}\usepackage{hyperref}
\usepackage[parfill]{parskip}
\usepackage[utf8]{inputenc}

\newcommand{\mt}{M2CPU}
\newcommand{\mi}{M2 instruction set}
\newcommand{\ma}{M2 architecture}

\title{M2 Architecture And Implementation}
\author{Zakary Nafziger}
\date{15 December 2017}

\begin{document}

\maketitle
\pagenumbering{gobble}
\newpage

\tableofcontents
\pagenumbering{gobble}
\newpage

\pagenumbering{arabic}

\section{Introduction}
The \mt{} is a simple subscalar 8-bit processor that implements the 75 
instruction \mi{}. The \mt{} and \mi{} were designed and implemented by the author
as a personal hobby project. The \ma{} is intended to be as simple as possible 
while still being interesting. Some basic information: 
\par

\begin{itemize}
\item 8-bit data and ALU buses
\item 16-bit address bus; 64KB of addressable memory
\item 256B zero-page stack
\item 75 instructions
\item implemented in VHDL on a MAX10 based development board
\end{itemize}

This document describes the \mt{}. The \mt{} is only one part of a larger 
project, the M2 computer. As of this writing the other major components of the
M2 computer project are an assembler and VGA video system. The assembler and 
video system are documented seperately. 
\par

\subsection{Terminology}
Throughout this document the terms \mt{}, \mi{}, and \ma{} will be used. \mt{} 
refers to the VHDL processor that implements the \mi{}. The \mi{} is the set of
75 instructions that make up the M2 assembly language. The \ma{} refers to the
combination of a processor and instruction set that implement the M2 assembly 
language. The \mt{} is therefore an implementation of the \ma{}, since it 
supports the complete \mi{}. However since the \mt{}'s instructions are 
implemented in microcode it could implement a different architecture. This 
document describes a complete implementation of the \ma{}, namely the \mt{}.
\par

The name 'M2' stands for 'Model 2' The 'Model 1' processor was a 4-bit system 
that was constructed by the Author. The Model 1 was implemented in discrete 
hardware, (about 1200 resistors and transistors). The Model 1 is documented 
here: 
\url{https://hackaday.io/project/665-4-bit-computer-built-from-discrete-transistors}
\par

\newpage
\section{Instruction Set}
\subsection{Instruction Table}
\subsection{Instruction Descriptions}

\newpage
\section{\mt Architecture}
\subsection{Block Diagram}
\subsection{Bus Architecture}
\subsection{Machine Cycle}

\newpage
\section{Microcode}
\subsection{Implementation}
\subsection{Detailed Microstates}

\newpage
\section{VHDL Implementation}

\newpage
\section{Conclusion}

\end{document}
